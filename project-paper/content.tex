% status: 40
% chapter: TBD

\title{Blockchain Implementation with Flask}


\author{Joao Paulo Leite}
\affiliation{%
  \institution{Indiana University}
  \streetaddress{Smith Research Center}
  \city{Bloomington} 
  \state{IN} 
  \postcode{47408}
  \country{USA}}
\email{jleite@ui.com}

% The default list of authors is too long for headers}
\renewcommand{\shortauthors}{J. p. Leite}


\begin{abstract}
The project is an implementation of Blockchain technology. The main purpose of 
this paper is to create a simple Blockchain implementation and simulate a 
network of nodes interact with the Blockchain. To accomplish this, all 
interaction with the Blockchain are brokered through an API deployed using 
Python Flask. The goal is to have multiple nodes running and interacting with 
the Blockchain to show how a distributed ledger system functions.

\end{abstract}

\keywords{hid-sp18-414, Blockchain, Distributed Ledger, Flask, Python}


\maketitle

\section{Introduction}

The concept of Blockchain technology was first introduced by Satoshi Nakamoto in
his whitepaper, “Bitcoin: A Peer-to-Peer Electronic Cash System” in early 2008. 
The concepts that were covered in his original paper is the guideline that was 
followed to build this Blockchain example. It is important to note that we have 
developed a simple example that is meant to be an introduction into how 
blockchains and distrubuted ledger technology functions. Although the 
technology has grown tremendously since 2008, this project will focus on the 
core technology behind Blockchain 1.0. Topics such as smart 
contracts(Blockchain 2.0) and Decentralized apps(Blockchain 3.0) are not part of 
this project but understanding the basic design principles behind Blockchain 1.0 
gives a good foundation to build upon. 

\section{Overview of Blockchain Technology}

Blockchain 1.0 is considered the first iteration of Blockchain Technology. 
Within this iteration, the goal was to eliminate the centralization and to 
create an open and inclusive global currency. To accomplish this goal, the 
blockchain is reliant on a distributed ledger which is stored and shared across 
the network. As records are added to the ledger, they become a permanent and 
unalterable part of the blockchain. To protect against tampering and bad actors, 
Consensus algorithms are deployed to ensure that the network is always in 
agreement as to legitimacy of the chain. The first of these algorithms to be developed (and 
the one deployed for this project) was the Proof of Work Algorithm. This 
algorithm relies on extremely difficult cryptographic puzzles(work) which can 
only be resolved through brute force. Within the network, different 
members(miners) will attempt to solve this puzzle for a reward. Once they solve 
the puzzle, they announce the solution to the network. Before the  nodes within the network 
adopt this new block, they will confirm that the solution provided did in fact solve the 
puzzle. Once it is confirmed,a new block is recorded on the chain and within this block, a new 
transactions can be stored. From a functional point of view, once the transactions have been 
placed in the blockchain, the transaction can be deemed complete.  


\section{Tools and Technology}

The tools and technology deployed for this project are going to be covered in 
this section.

\subsection{Flask - Web Framework}

Flask is a micro-framework for Python based on Werkzeug and Jinga 2. The reason 
it is called a micro framework is because it doesn't require and specific tools 
or libraries. For this project, Flask was used to build the REST API which 
allows us to interact and manipulate the blockchain. 

Code Example:
\begin{verbatim}
  from flask import request, Flask, jsonify
  app = Flask(__name__)

  @app.route('/mine', methods=['GET'])
  def mine():
  	
\end{verbatim} 

Install:
\begin{verbatim}
  pip install flask
\end{verbatim}

Console Example:
\begin{verbatim}
   * Running on http://0.0.0.0:9999/ (Press CTRL+C to quit)
\end{verbatim}

\subsection{Hashlib}

Hashlib is a Python module with a common interface to many of the more widely used hash and message digest algorithms.For this project, it was used to call the SHA256 hashing function.

Code Example:
\begin{verbatim}
	def hashing(block):

   		return hashlib.sha256(json.dumps(block, sort_keys=True).encode()).hexdigest()
\end{verbatim} 

Install:
\begin{verbatim}
  easy_install Hashlib
\end{verbatim}


\subsection{Python}

Python is the high-level programming language that was used to develop this project.

\subsection{Requests}

Requests is a Python module that allows you to craft HTTP/1.1 requests without needing to 
include the query string or to format the POST data.

Code Example:
\begin{verbatim}
	response = requests.get(f'http://{node}/chain')
\end{verbatim} 

Install:
\begin{verbatim}
  pip install Requests
\end{verbatim}


\subsection{UUID – Universally unique identifiers}

UUID is a Python module that is used to create unique identifiers without the need for a 
centralized server/registrar. The values created by UUID are 128 bits long and work well with 
cryptographic hashing which provided useful when assigned node identifiers.

Code Example:
\begin{verbatim}
	nodeid = str(uuid1()).replace('-', '')
\end{verbatim} 

Install:
\begin{verbatim}
  pip install uuid
\end{verbatim}



\section{Conclusion}

Put here an conclusion. Conclusion and abstracts must not have any
citations in the section.


\begin{acks}

  The authors would like to thank Dr.~Gregor~von~Laszewski for his
  support and suggestions to write this paper.

\end{acks}

\bibliographystyle{ACM-Reference-Format}
\bibliography{report} 

