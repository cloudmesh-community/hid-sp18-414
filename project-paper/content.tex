% status: 95
% chapter: Blockchain

\title{Blockchain Implementation with Flask}


\author{Joao Paulo Leite}
\affiliation{%
  \institution{Indiana University}
  \streetaddress{Smith Research Center}
  \city{Bloomington} 
  \state{IN} 
  \postcode{47408}
  \country{USA}}
\email{jleite@ui.com}

% The default list of authors is too long for headers}
\renewcommand{\shortauthors}{J. P. Leite}


\begin{abstract}
  The main purpose of this paper is to create a simple Blockchain
  implementation and simulate a network of nodes interacting with the
  Blockchain. To accomplish this, all interaction with the Blockchain
  are brokered through an API deployed using Python Flask. The goal is
  to have multiple nodes running and interacting with the Blockchain
  to show how a distributed ledger system functions.

\end{abstract}

\keywords{hid-sp18-414, Blockchain, Distributed Ledger, Flask, Python}


\maketitle

\section{Introduction}

The concept of blockchain technology was first introduced by Satoshi
Nakamoto in his whitepaper, ``Bitcoin: A Peer-to-Peer Electronic Cash
System'' in early
2008~\cite{hid-sp18-414-www-blockchain-theory-application}.  The
concepts that were outlined from Satwik Kansal's tutorial was guideline that
was followed to build this simple blockchain example~\cite{hid-sp18-414-www-blockchain-example}. 
It is important to notethat we have developed a basic example that is meant to be an
introduction into how blockchains and distrubuted ledger technology
functions. Although the technology has grown tremendously since 2008,
this project will focus on the core technology behind Blockchain
1.0. Topics such as smart contracts(Blockchain 2.0) and Decentralized
apps(Blockchain 3.0) are not part of this project but understanding
the basic design principles behind Blockchain 1.0 gives a good
foundation to build upon.

\section{Overview of Blockchain Technology}

Blockchain 1.0 is considered the first iteration of blockchain
technology.  With this iteration, the goal was to eliminate the
centralization and to create an open and inclusive global currency. To
accomplish this goal, the blockchain is reliant on a distributed
ledger which is stored and shared across the
network~\cite{hid-sp18-414-www-promise-bitcoin-blockchain}.  As
records(blocks) are added to the ledger, they become a permanent and
unalterable part of the blockchain. To protect against tampering and
bad actors, Consensus algorithms are deployed to ensure that the
network is always in agreement as to legitimacy of the chain. The
first of these algorithms to be developed (and the one deployed for
this project) was the Proof of Work Algorithm. This algorithm relies
on extremely difficult cryptographic puzzles(work) which can only be
resolved through brute force. Within the network, different
members(miners) will attempt to solve this puzzle for a reward. Once
they solve the puzzle, they announce the solution to the
network. Before any node within the network adopts this new block,
they will confirm that the solution provided did in fact solve the
puzzle. Once it is confirmed,a new block is recorded on the chain and
within this block, new transactions can be stored. From a functional
point of view, once a transaction has been placed in the Blockchain,
the transaction can be deemed
complete.~\cite{hid-sp18-414-financialinnovation-zhao}

\section{API}

This project utilizes Python Flask to expose some standard operations
that are common with any distributed ledger. The two most fundamental
functions to any blockchain are the posting of transactions and the
mining of new blocks. These functions go hand in hand, as new
transactions can only be posted to a block when a new block is mined
and added to the chain. Some secondary functions that provide
stability and agreement between all the nodes in the network is the
Consensus algorithm and the discovery of other nodes in the network
through Node Registration. Between these four functions, the project
is able to product a functioning Blockchain where multiple nodes can
interact by posting transactions and mining new blocks, while staying
in sync and maintaining the ledger's integrity.

\subsection{Consensus}

The Consensus command is implemented to provide a mechanism to
maintain the Blockchain uniformly across all the nodes within the
network. In this project, the consensus is resolved by analyzing which
Blockchain across all the nodes is the largest.

\bigskip
\noindent
\textbf{Consensus code which analyzes the current nodes length vs the other nodes in the
network:}
\begin{footnotesize}
\begin{verbatim}

    def consensus(self):
        ...
        len_chain = len(self.chain)
        for node in others:
            strurl = f'http://{node}/chain'
            response = requests.get(strurl)
                if response.status_code == 200:
                	chain = response.json()['chain']
                    length = response.json()['len']

                    if length > len_chain
            
\end{verbatim}
\end{footnotesize}
When one node successfully mines a new block, they broadcast their
solution so that the other nodes can begin the adoption process. The
other nodes are then responsible to come to a consensus around the
legitimacy of the new block. These nodes can quickly validate that the
solution(proof) provided does maintain the integrity of the chain by
using the solution to solve the cryptographic puzzle.

\bigskip
\noindent
\textbf{Validation code that checks the proposed block's solution
(\textbf{block['proof']}) as valid:}
\begin{footnotesize}
\begin{verbatim}

    def validation_chain(self, chain):
        ...
        if not self.validation_block(last_block,block,last_hash):
            return False

        last_block = block
        current_index += 1

        return True
	
\end{verbatim}
\end{footnotesize}
\bigskip
\noindent
As each node confirms the solution provided, the block is adopted by that node.

\bigskip
\noindent
\textbf{Consensus code which adopts the new blockchain 
as it's own:}
\begin{footnotesize}
\begin{verbatim}

    def consensus(self):
        ...
	    if replacement_chain:
            self.chain = replacement_chain
            return True
        return False
	
\end{verbatim}
\end{footnotesize}

If the node is a miner node, once the block has been adopted, all the
work done to find the solution is lost and must be restarted with the
information of the adopted block.

\subsection{Transactions}

The transaction command provides the ability for a node to issue a
transaction which will be published on the blockchain. The transaction
function is simplified in this project, consisting only of a sender
address, a recipient address and an amount. In a production level
scenario, transactions may include a transaction fee which can be set
at the time the transaction is created. This is because each block has
a max size which limits the number of transactions that can be posted
within a block. Because of this fact, all transactions go to a pool
(TX pool) and the transactions offering a higher fees are prioritized.

\bigskip
\noindent
\textbf{Create Message code:}
\begin{footnotesize}
\begin{verbatim}

    def create_message(self, Timestamp, Sender, Message):
        ...
        self.trans.append({
            'Timestamp': Timestamp,
            'Sender': Sender,
            'Message': Message,
        })
        
\end{verbatim}
\end{footnotesize}

\subsection{Node Registration}

The registration command provides the ability for a node to register
other known nodes. In this project, a node is represented by
URL (ex: \verb|http://127.0.0.1:8888|). The nodes are stored in a set, to be
used by the consensus command to access other node's chains.

\bigskip
\noindent
\textbf{Node Registration code:}
\begin{footnotesize}
\begin{verbatim}

    def node_registration(self, address):

        url = urlparse(address)
        if url.netloc:
            self.nodes.add(url.netloc)
        elif url.path:
            self.nodes.add(url.path)
        else:
            print("ERROR")
        
\end{verbatim}
\end{footnotesize}


\bigskip
\noindent
\textbf{Consensus algorithm leveraging stored nodes:}
\begin{footnotesize}
\begin{verbatim}

    def consensus(self):
        ....
        others = self.nodes
        
        ....
        for node in others:
            strurl = f'http://{node}/chain'
            response = requests.get(strurl)
        
\end{verbatim}
\end{footnotesize}

\subsection{Mine}

The mine command allows for new blocks to be mined for the chain. In
this project, the difficulty threshold for the proof of work algorithm
was set very low so that blocks can be created almost instantaneously.

\bigskip
\noindent
\textbf{Proof of Work algorithm:}
\begin{footnotesize}
\begin{verbatim}

  def pow(self, oldBlock):
      'Proof of Work Algorithm'

      oldproof = oldBlock['proof']
      oldhash = self.hashing(oldBlock)
      newproof = 0

      while self.validation_block(oldproof,newproof,oldhash) is False:
            newproof += 1

        return newproof
        
\end{verbatim}
\end{footnotesize}

Typically in a production scenario, the goal is to have the
difficultly threshold set to a level which causes a block to be
created every 10 minutes(on average). Because the mining pool's hash
power can fluctuate, the difficulty threshold is adjusted to ensure it
does not become to easy or difficult in relation to the hash
power.~\cite{hid-sp18-414-www-pow-vs-pos}

\bigskip
\noindent
\textbf{Validation Block code which is the main component of the PoW algorithm:}
\begin{footnotesize}
\begin{verbatim}

    def validation_block(oldproof, proof, oldhash):
        'Validates the solution to the PoW'

        encoding = (str(oldproof)+str(proof)+str(oldhash)).encode()
        attempt = hashlib.sha224(encoding).hexdigest()
        return attempt[:POW_Difficulty] == '0' * POW_Difficulty

        
\end{verbatim}
\end{footnotesize}

To adjust the difficulty in this project, we can increase the number
of consecutive zero's needed to be returned from the
hash. Alternatively, if we decrease the number of consecutive zero's,
the difficulty is reduced.

\section{Examples}

\subsection{register}

\bigskip
\noindent
\begin{footnotesize}
\begin{verbatim}

request:
curl -H "Content-Type: application/json"  \
        -X POST \
        -d '{"nodes": ["http://127.0.0.1:8888"]}' \
        http\://localhost\:9999/register

response:
{
  "message": "Nodes Added",
  "nodes": [
    "127.0.0.1:8888"
  ]
}

\end{verbatim}
\end{footnotesize}

\subsection{consensus}

\bigskip
\noindent
\begin{footnotesize}
\begin{verbatim}

request:
curl -H "Content-Type: application/json" \
        -X GET \
        http\://localhost\:8888/consensus

response:
{
  "message": "Chain not Replaced",
  "replacement_chain": [
    {
      "TS": 1525148850.3083794,
      "index": 1,
      "oldhash": "1",
      "proof": 100,
      "trans": []
    }
  ]
}

\end{verbatim}
\end{footnotesize}

\subsection{newtransaction}

\bigskip
\noindent
\begin{footnotesize}
\begin{verbatim}

request:
curl -H "Content-Type: application/json"  \
    -X POST \
    -d '{"Sender": "John", "Message": "Hello"}' \
    http\://localhost\:8888/newtransaction
	

response:
{
  "TS": 1525463137.4779193,
  "index": 2,
  "oldhash": "66ae3106045d3e00f8d62204a3e269381cd91deb6df574b067c125d9",
  "proof": 213
  "trans": [
    {
     "Message": "Hello",
     "Sender": "John",
     "Timestamp": 1525463137.0406077
     }
 }

\end{verbatim}
\end{footnotesize}

\subsection{chain}

\bigskip
\noindent
\begin{footnotesize}
\begin{verbatim}

request:
curl -H "Content-Type: application/json" \
        -X GET \
        http\://localhost\:8888/chain
        
response:
{
  "chain": [
    {
      "TS": 1525148850.3083794,
      "index": 1,
      "oldhash": "1",
      "proof": 100,
      "trans": []
    }
  ],
  "len": 1
}

\end{verbatim}
\end{footnotesize}

\subsection{mine}

\bigskip
\noindent
\begin{footnotesize}
\begin{verbatim}

request:
curl -H "Content-Type: application/json" \
        -X GET \
        http\://localhost\:9999/mine
        
response:
{
  "block number": 3,
  "message": "New Block Created",
  "previous hash": "5e67bb7aec15844ab65ebc890ff421eddd14eaa7aaca86009225d20c",
  "proof": 1758,
  "trans": [
    {
      "Message": "Hello",
      "Sender": "Luciana",
      "Timestamp": 1525463138.9406605
    },
    {
      "Message": "How are you",
      "Sender": "Luciana",
      "Timestamp": 1525463139.3841374
    }
  ]
}


\end{verbatim}
\end{footnotesize}

\section{Tools and Technology}

The tools and technology deployed for this project are going to be covered in 
this section.

\subsection{Flask - Web Framework}

Flask is a micro-framework for Python based on Werkzeug and Jinga
2. The reason it is called a micro framework is because it doesn't
require any specific tools or libraries. For this project, Flask was
used to build the REST API which allows us to interact with specific
functions of the
blockchain~\cite{hid-sp18-414-www-flask-python-microframewor}.

\bigskip
\noindent
\textbf{Code Example:}
\begin{footnotesize}
\begin{verbatim}

    from flask import request, Flask, jsonify
   	app = Flask(__name__)

    @app.route('/mine', methods=['GET'])
    def mine():
    
\end{verbatim}
\end{footnotesize}
\noindent
\textbf{Install:}
\begin{footnotesize}
\begin{verbatim}

    pip install flask
    
\end{verbatim}
\end{footnotesize}
\noindent
\textbf{Console Example:}
\begin{footnotesize}
\begin{verbatim}

    * Running on http://0.0.0.0:9999/ (Press CTRL+C to quit)
    
\end{verbatim}
\end{footnotesize}
\subsection{Hashlib}

Hashlib is a Python module with a common interface to many of the more widely used hash and 
message digest algorithms.For this project, it was used to call the SHA256 hashing function~\cite{hid-sp18-414-www-hashlib-secure}.

\bigskip
\noindent
\textbf{Code Example:}
\begin{footnotesize}
\begin{verbatim}

    def hashing(block):

    return hashlib.sha256(...).hexdigest()
    
\end{verbatim}
\end{footnotesize}
\noindent
\textbf{Install:}
\begin{footnotesize}
\begin{verbatim}

    easy_install Hashlib
    
\end{verbatim}
\end{footnotesize}

\subsection{Python}

Python is the high-level programming language that was used to develop
this project.

\subsection{Requests}

Requests is a Python module that allows you to craft HTTP/1.1 requests
without needing to include the query string or to format the POST
data~\cite{hid-sp18-414-www-requests-HTTP}.

\bigskip
\noindent
\textbf{Code Example:}
\begin{footnotesize}
\begin{verbatim}
    response = requests.get(f'http://{node}/chain')
\end{verbatim}
\end{footnotesize}
\noindent
\textbf{Install:}
\begin{footnotesize}
\begin{verbatim}
    pip install Requests
\end{verbatim}
\end{footnotesize}

\subsection{Universally unique identifiers}

Universally unique identifiers(UUID) is a Python module that is used
to create unique identifiers without the need for a centralized
server/registrar. The values created by UUID are 128 bits long and
work well with cryptographic hashing which provided useful when
assigned node identifiers~\cite{hid-sp18-414-www-uuid-objects}.

\bigskip
\noindent
\textbf{Code Example:}
\begin{footnotesize}
\begin{verbatim}
    nodeid = str(uuid1()).replace('-', "")
\end{verbatim}
\end{footnotesize}
\noindent
\textbf{Install:}
\begin{footnotesize}
\begin{verbatim}
    pip install uuid
\end{verbatim}
\end{footnotesize}

\section{Conclusion}

This project's aim was to provide a blockchain implementation with all
the basic functionality needed to deploy it in a multi-node
scenario. The project was able to demonstrate a blockchain that could
add transactions, create new blocks to the chain and maintain the
integrity of the chain across all the nodes on the network. This was
accomplished using Python Flask as the backbone to create and
deploying nodes.

\begin{acks}

  The authors would like to thank Dr.~Gregor~von~Laszewski for his
  support and suggestions to write this paper.

\end{acks}

\bibliographystyle{ACM-Reference-Format}
\bibliography{report} 

